\documentclass[14pt]{extarticle} 
\usepackage{amsmath,mathtools,amsfonts,amsthm,amssymb,hyperref}
\usepackage{wasysym,geometry,bussproofs,latexsym,parskip,bookmark}
\usepackage{mathtools,float}
\newtheorem{defn}{Definition}
\newtheorem{thm}{Theorem}
\newtheorem{claim}{Claim}
\newtheorem{lemma}{Lemma}
\newcommand{\dps}{\displaystyle}
\hypersetup{colorlinks,allcolors=blue,linktoc=all}
\geometry{a4paper} 
\geometry{margin=0.5in}
\title{Math for CS 2015/2019 solutions to ``In-Class Problems Week 11, Wed. (Session 26)''}
\author{https://github.com/spamegg1}
\begin{document}
\maketitle
\tableofcontents

\section{Problem 1}
Your class tutorial has 12 students, who are supposed to break up into 4 groups of 3 students each. Your Teaching Assistant (TA) has observed that the students waste too much time trying to form balanced groups, so he decided to pre-assign students to groups and email the group assignments to his students.

\subsection{(a)}
Your TA has a list of the 12 students in front of him, so he divides the list into consecutive groups of 3. For example, if the list is ABCDEFGHIJKL, the TA would define a sequence of four groups to be $(\{A,B,C\},\{D,E,F\},\{G,H,I\},\{J,K,L\})$. This way of forming groups defines a mapping from a list of twelve students to a sequence of four groups. This is a $k$-to-1 mapping for what $k$?
\begin{proof}
Two lists map to the same sequence of groups iff the first 3 students are the same on both lists, and likewise for the second, third, and fourth consecutive sublists of 3 students. So for a given sequence of 4 groups, the number of lists which map to it is
$(3!)^4$ because there are $3!$ ways to order the students in each of the 4 consecutive sublists.
\end{proof}

\subsection{(b)}
A group assignment specifies which students are in the same group, but not any order in which the groups should be listed. If we map a sequence of 4 groups,
$$
(\{A,B,C\},\{D,E,F\},\{G,H,I\},\{J,K,L\})
$$
into a group assignment
$$
\{\{A,B,C\},\{D,E,F\},\{G,H,I\},\{J,K,L\}\}
$$
the mapping is $j$-to-1 for what $j$?
\begin{proof}
$j = 4!$. Each of the $4!$ sequences of a particular set of four groups maps to that set of groups.
\end{proof}

\subsection{(c)}
How many group assignments are possible?
\begin{proof}
$\dps\frac{12!}{4!\cdot (3!)^4} = 15400$ different assignments.

There are $12!$ possible lists of students, and we can map each list to an assignment by first mapping the list to a sequence of four groups, and then mapping the sequence to the assignment. Since the first map is $(3!)^4$-to-1 and the second is $4!$-to-1, the composite map is $(3!)^4\cdot 4!$-to-1. So by the Division Rule, $12! = ((3!)^4 \cdot 4!) A$ where $A$ is the number of assignments.
\end{proof}

\subsection{(d)}
In how many ways can $3n$ students be broken up into $n$ groups of 3?
\begin{proof}
$\dps\frac{(3n)!}{(3!)^n \cdot n!}$. This follows simply by replacing “12” by “$3n$” and “4” by “$n$” in the solution to the previous problem parts.
\end{proof}

\section{Problem 2}
\subsection{(a)}
There are 30 books arranged in a row on a shelf. In how many ways can eight of these books be selected so that there are at least two unselected books between any two selected books?
\begin{proof}
Consider a binary string where ones represent the selected books. Here's one example with 4 unselected books at left and right ends:
$$
000010010010010010010010010000
$$
If any two selected books (ones) have at least two unselected books (zeroes) between them, there are at least 14 zeroes that are between the ones. Adding the 8 ones, we get 22 binary digits. 

This leaves 30 - 22 = 8 more zeroes that can be interspersed anywhere. For example we could put all 8 zeros on the left:
$$
000000001001001001001001001001
$$
or all on the right:
$$
100100100100100100100100000000
$$
or spread them in between the ``100'' blocks:
$$
100\,\,0\,\,100\,\,0\,\,100\,\,0\,\,100\,\,0\,\,100\,\,0\,\,100\,\,0\,\,100\,\,0\,\,1\,\,0
$$
These are just a few examples to get the idea across.

Consider the length 16 binary strings, that are obtained from the strings we want (above) by deleting the 14 zeroes in between the ones. Here's one example:
$$
00001\underline{00}1\underline{00}1\underline{00}1\underline{00}1\underline{00}1\underline{00}1\underline{00}10000
$$
$$
00001\,\,1\,\,1\,\,1\,\,1\,\,1\,\,1\,\,10000
$$
In these length 16 strings we can place the 8 ones and the remaining 8 zeroes in any way we like. (The deleted zeroes are ``there'' invisibly, taking care of the ``there must be at least two unselected books between any two selected books'' rule for us.)

So there is a bijection between the book selections described in the problem, and the set of length 16 binary strings with 8 ones.

Therefore the answer is $\dps\binom{16}{8}$.
\end{proof}

\subsection{(b)}
How many nonnegative integer solutions are there for the following equality?
$$
x_1 + x_2 + \ldots + x_m = k
$$
\begin{proof}
There is a bijection between the set of solutions to this equality
$$
\{(x_1, \ldots, x_m) \in \mathbb{N}^m \,\,\,|\,\,\,x_1 + \ldots + x_m = k\}
$$ 
and the set of binary strings with $k$ zeroes and $m-1$ ones. Let $0^x$ denote a sequence of $x$ zeroes. Here is the bijection:
$$
(x_1, \ldots, x_m) \longleftrightarrow 0^{x_1} \,1\, \ldots 0^{x_{m-1}} \,1\, 0^{x_m}
$$
Here there are $k$ zeroes because $x_1 + x_2 + \ldots + x_m = k$, and there are $m-1$ ones because the ones are separators between the $m$ sequences of zeroes.

The number of such binary strings is $\dps\binom{k+m-1}{m-1}$.
\end{proof}

\subsection{(c)}
How many nonnegative integer solutions are there for the following inequality?
$$
x_1 + x_2 + \ldots + x_m \leq k
$$
\begin{proof}
Remember that in the In-Class Problems 25, we described a bijection between the set $S_{k,m}$ of solutions to this inequality, and the set of binary strings with $k$ zeroes and $m$ ones. So we simply need to count that set of binary strings.

The number of such binary strings is $\dps\binom{k+m}{m}$.
\end{proof}

\subsection{(d)}
How many length $m$ weakly increasing sequences of nonnegative integers $\leq k$ are there?
\begin{proof}
Remember that in the In-Class Problems 25, we described a bijection between the set $S_{k,m}$ of solutions to the inequality in part (c), and the set $L_{k,m}$ of length $m$ weakly increasing sequences of nonnegative integers $\leq k$.

So the answer is the same as part (c): $\dps\binom{k+m}{m}$.
\end{proof}

\section{Problem 3}
The Tao of BOOKKEEPER: we seek enlightenment through contemplation of the word $BOOKKEEPER$.

\subsection{(a)}
In how many ways can you arrange the letters in the word $POKE$?
\begin{proof}
There are $4!$ arrangements corresponding to the $4!$ permutations of the set $\{P, O, K, E\}$.
\end{proof}

\subsection{(b)}
In how many ways can you arrange the letters in the word $BO_1 O_2 K$? Observe that we have subscripted the O’s to make them distinct symbols.
\begin{proof}
There are $4!$ arrangements corresponding to the $4!$ permutations of the set $\{B, O_1 , O_2 , K\}$.
\end{proof}

\subsection{(c)}
Suppose we map arrangements of the letters in $BO_1 O_2 K$ to arrangements of the letters in $BOOK$ by erasing the subscripts. Indicate with arrows how the arrangements on the left are mapped to the arrangements on the right.
$$
\begin{array}{cccc}
O_2 B O_1 K & & &\\
KO_2 BO_1   & & &\\
O_1 BO_2 K  & & &BOOK\\
KO_1 BO_2   & & &OBOK\\
BO_1 O_2 K  & & &KOBO\\
BO_2 O_1 K  & & &\ldots\\
\ldots      & & &
\end{array}
$$

\begin{proof}
Lines 1 and 3 on the left map to $OBOK$, lines 2 and 4 map to $KOBO$, lines 5 and 6 map to $BOOK$.
\end{proof}

\subsection{(d)}
What kind of mapping is this, young grasshopper?
\begin{proof}
2-to-1.
\end{proof}

\subsection{(e)}
In light of the Division Rule, how many arrangements are there of $BOOK$?
\begin{proof}
$4! / 2$
\end{proof}

\subsection{(f)}
Very good, young master! How many arrangements are there of the letters in\\ $KE_1 E_2 PE_3 R$?
\begin{proof}
$6!$
\end{proof}

\subsection{(g)}
Suppose we map each arrangement of $KE_1 E_2 PE_3 R$ to an arrangement of $KEEPER$ by erasing subscripts. List all the different arrangements of $KE_1 E_2 PE_3 R$ that are mapped to $REPEEK$ in this way.
\begin{proof}
$$
\begin{array}{c}
RE_1 P E_2 E_3 K\\ 
RE_1 P E_3 E_2 K\\ 
RE_2 P E_1 E_3 K\\ 
RE_2 P E_3 E_1 K\\ 
RE_3 P E_1 E_2 K\\ 
RE_3 P E_2 E_1 K
\end{array}
$$
\end{proof}

\subsection{(h)}
What kind of mapping is this?
\begin{proof}
$3! = 6$, so it's 6-to-1.
\end{proof}

\subsection{(i)}
So how many arrangements are there of the letters in $KEEPER$?
\begin{proof}
$6!/3!$
\end{proof}

{\it Now you are ready to face the BOOKKEEPER!}

\subsection{(j)}
How many arrangements of $BO_1 O_2 K_1 K_2 E_1 E_2 PE_3 R$ are there?
\begin{proof}
$(10)!$
\end{proof}

\subsection{(k)}
How many arrangements of $BOOK_1 K_2 E_1 E_2 PE_3 R$ are there?
\begin{proof}
$(10)! / 2!$
\end{proof}

\subsection{(l)}
How many arrangements of $BOOKKE_1 E_2 PE_3 R$ are there?
\begin{proof}
$(10)! / (2! \cdot 2!)$
\end{proof}

\subsection{(m)}
How many arrangements of $BOOKKEEPER$ are there?
\begin{proof}
$$
\binom{10}{1,2,2,3,1,1} \Coloneqq \frac{(10)!}{1!2!2!3!1!1!} = \frac{(10)!}{(2!)^2 \cdot 3!}
$$
\end{proof}

\begin{center}
{\it Remember well what you have learned: subscripts on, subscripts off.

This is the Tao of Bookkeeper.}
\end{center}

\subsection{(n)}
How many arrangements of $VOODOODOLL$ are there?
\begin{proof}
$$
\binom{10}{1,2,5,2} \Coloneqq \frac{(10)!}{1!2!5!2!}
$$
\end{proof}

\subsection{(o)}
How many length 52 sequences of digits contain exactly 17 two’s, 23 fives, and 12 nines?
\begin{proof}
$$
\binom{52}{17,23,12} \Coloneqq \frac{(52)!}{(17)!(23)!(12)!}
$$
\end{proof}

\section{Problem 4}
Find the coefficients of

\subsection{(a)}
$x^5$ in $(1 + x)^{11}$
\begin{proof}
By the Binomial Theorem $\dps (1 + x)^{11} = \sum_{i = 0}^{11} \binom{11}{i}x^i$. So $x^5$ has coefficient $\binom{11}{5}$.
\end{proof}

\subsection{(b)}
$x^8y^9$ in $(3x+2y)^{17}$
\begin{proof}
By the Binomial Theorem 
$$
(3x + 2y)^{17} = \sum_{i = 0}^{17} \binom{17}{i}(3x)^i (2y)^{17-i}
$$ 
$x^8y^9$ corresponds to $i = 8$. So it has coefficient $\binom{17}{8}\cdot 3^8 \cdot 2^9$.
\end{proof}

\subsection{(c)}
$a^6b^6$ in $(a^2+b^3)^5$
\begin{proof}
By the Binomial Theorem 
$$
(a^2 + b^3)^{5} = \sum_{i = 0}^{5} \binom{5}{i}(a^2)^i (b^3)^{5-i}
$$ 
$a^6b^6$ corresponds to $i = 3$. So it has coefficient $\binom{5}{3}$.
\end{proof}

\section{Problem 5}
\subsection{(a)}
An independent living group is hosting nine new candidates for membership. Each candidate must be assigned a task: 1 must wash pots, 2 must clean the kitchen, 3 must clean the bathrooms, 1 must clean the common area, and 2 must serve dinner. Write a multinomial coefficient for the number of ways this can be done.
\begin{proof}
There is a bijection from sequences containing one $P$, two $K$’s, three $B$’s, a $C$, and two $D$’s. In any such sequence, the letter in the ith position specifies the task assigned to the $i$th candidate. Therefore, the number of possible assignments is:
$$
\binom{9}{1,2,3,1,2} \Coloneqq \frac{9!}{1!2!3!1!2!}
$$
\end{proof}

\subsection{(b)}
How many nonnegative integers less than 1,000,000 have exactly one digit equal to 9 and have a sum of digits equal to 17?
\begin{proof}
We identify the nonnegative integers less than 1,000,000 with the length 6 strings of decimal digits. Then there is a bijection with pairs:

\begin{center}
(position of the 9, successive values of other 5 digits)
\end{center}

The sum of the other 5 digits is equal to 8, so the number of ways to choose their values is equal to the number of solutions over the nonnegative integers to (1), namely, $\binom{12}{4}$. So by the product rule there are $6\cdot \binom{12}{4}$ such integers.
\end{proof}
\end{document}
